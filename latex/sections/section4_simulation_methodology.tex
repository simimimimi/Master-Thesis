\section{Simulation Methodology} \label{sec:SimulationMethodology}

This section provides a detailed description of the simulation framework used to generate asset price paths under the Hypergeometric Volatility Model. First, the latent volatility signal is simulated. Then, the asset price dynamics are determined by sampling the log-returns based on the generated volatility. Finally, European option prices are computed via Monte Carlo simulation, and implied volatilities are derived through inversion of the Black-Scholes formula.


\subsection{Time Discretization} \label{subsec:TimeDiscretization}

The model presented in Section~\ref{sec:ModelingFramework} is defined in continuous time. However, for the purpose of simulation, a discrete-time version is considered. Let $T$ denote the time horizon and define an equidistant time grid $0 = t_0 < t_1 < \ldots < t_N = T$ with step size $\Delta t = T/N$. At each time step $t_i$ for $i = 0,\ldots,N-1$, Brownian motion increments are generated, and then the state of the signal and asset price processes are updated.

The simulation requires two Brownian motions: $B$, which drives the volatility signal, and $W$, which drives the asset price. These are modeled as correlated processes to reflect the leverage effect commonly observed in financial markets. At each time step $t_i$, correlated Brownian motion increments $(\Delta B_i, \Delta W_i)$ are generated such that $\mathrm{Cov}(\Delta B_{t_i}, \Delta W_{t_i}) = \rho \cdot \Delta t$.

This is achieved by drawing two independent standard normal random variables $\varepsilon_i$ and $\eta_i$ at each time step. Then the discrete Brownian increments are
\begin{align} \label{eq:BrownianMotions}
    \Delta B_{t_i} &= \sqrt{\Delta t} \cdot \varepsilon_i, \\
    \Delta W_{t_i} &= \rho \cdot \Delta B_{t_i} + \sqrt{1 - \rho^2} \cdot \sqrt{\Delta t} \cdot \eta_i.
\end{align}
This is repeated at each of the $N$ time points.


\subsection{Simulation of the Signal Process} \label{subsec:SimSignalProcess}

The latent volatility signal is modeled as a finite-dimensional approximation of an infinite mixture of Ornstein-Uhlenbeck processes, as described in Section~\ref{subsec:FiniteDimApprox}. For simulation purposes, the explicit solution of the Ornstein-Uhlenbeck stochastic differential equation is used. This method ensures numerical stability and avoids discretization bias.

Let $J$ denote the number of Ornstein-Uhlenbeck components, with mean-reversion speeds $\kappa_j$ and weights $c_j$ obtained from the geometric partitioning scheme. For each component $Z_t^{(j)}$, the update rule is given by
\begin{equation} \label{eq:TimeDiscrOUExact}
    Z_{t_{i+1}}^{(j)} = e^{-\kappa_j \Delta t} Z_{t_i}^{(j)} + \sqrt{\frac{1 - e^{-2 \kappa_j \Delta t}}{2 \kappa_j}} \cdot \frac{\Delta B_t}{\sqrt{\Delta t}},
\end{equation}
where $\Delta B_t$ corresponds to the volatility-driving Brownian motion across all $j = 1,\ldots,J$.

After updating all $Z^{(j)}$, the signal process is computed at each time step as a linear combination:
\begin{equation} \label{eq:TimeDiscrSignal}
    X_{t_i}^{(J)} = \sum_{j=1}^J c_j Z_{t_i}^{(j)}.
\end{equation}


\subsection{Volatility Specification} \label{subsec:VolaSpecification}

Once the signal process $X_t$ has been generated, the stochastic volatility process is obtained through the exponential transformation
\begin{equation} \label{eq:TimeDiscrVolatility}
    \lambda(X_{t_i}) = b \cdot \exp\left( a \cdot \frac{X_{t_i}}{\Bar{\sigma}_X} - \frac{a^2}{2} \right),
\end{equation}
where $b > 0$ controls the volatility level and $a > 0$ determines the sensitivity to changes in the signal. This specification ensures that the volatility remains strictly positive and allows for extreme spikes, in line with empirical observations in financial markets. To ensure comparability across different parameter settings, the signal is normalized by its long-term standard deviation $\Bar{\sigma}_X$ and centered by the correction term, as discussed in Section~\ref{subsec:LongTermVariance}.

The entire volatility path ${\lambda(X_{t_i})}_{i=0}^N$ is computed prior to asset price simulation and stored for later use in simulating the asset price process.


\subsection{Simulation of the Asset Price Process} \label{subsec:SimAssetPriceProcess}

Given the discrete-time volatility path and the correlated Brownian motion increments from Equation~\eqref{eq:BrownianMotions}, the asset price process is simulated using a log-Euler scheme under the risk-neutral measure. The log-return over each time step is given by
\begin{equation} \label{eq:TimeDiscrLogReturn}
    \log\left( \frac{S_{t_{i+1}}}{S_{t_i}} \right) = \left( r - \frac{1}{2} \lambda(X_{t_i})^2 \right) \Delta t + \lambda(X_{t_i}) \cdot \Delta W_{t_i},
\end{equation}
where $r$ denotes the constant risk-free interest rate. This recursion ensures that the asset price remains strictly positive and reflects the impact of the stochastic volatility.

To compute the entire path of the asset price process, log-returns are accumulated as follows
\begin{equation} \label{eq:TimeDiscrLogPrice}
    \log S_{t_{i+1}} = \log S_{t_i} + \left( r - \frac{1}{2} \lambda(X_{t_i})^2 \right) \Delta t + \lambda(X_{t_i}) \cdot \Delta W_{t_i},
\end{equation}
starting from an initial price $S_0 > 0$. The final price path is then recovered by exponentiating the log-values.

This procedure results in a full discrete-time asset price path $S_0, S_1, \ldots, S_N$. The simulation guarantees that the asset price remains strictly positive at all times.


\subsection{Monte Carlo Option Pricing} \label{subsec:MonteCarloOptionPricing}

The simulated asset price paths are used to compute European option prices via risk-neutral Monte Carlo valuation. Let $M$ denote the number of independent Monte Carlo paths. For each simulation run $m = 1, \ldots, M$, the price points corresponding to a pre-specified set of maturities $\{T_1,\ldots,T_Q\} \subset \{t_0,\ldots,t_N\}$ are extracted from the full trajectory $\{S_{t_i}^{(m)}\}_{i=0}^N$. This allows for efficient pricing of European options across multiple maturities without resimulating asset paths.

For each maturity $T_q$ and strike $K_l$ from the predefined grid $\{K_1,\ldots,K_L\}$, the discounted expected payoff of a European call option is estimated as
\begin{equation}
    C(K_l, T_q) = \mathbb{E}\left[ e^{-rT_q} (S_{T_q} - K_l)^+ \right] \approx \frac{e^{-rT_q}}{M} \sum_{m=1}^M \max\left(S_{T_q}^{(m)} - K_l, 0\right),
\end{equation}
and analogously, the European put price is given by
\begin{equation}
    P(K_l, T_q) = \mathbb{E}\left[ e^{-rT_q} (K_l - S_{T_q})^+ \right] \approx \frac{e^{-rT_q}}{M} \sum_{m=1}^M \max\left(K_l - S_{T_q}^{(m)}, 0\right).
\end{equation}
This procedure yields a matrix of option prices for each combination of strike and maturity. All pricing is performed under the risk-neutral measure, using the constant risk-free rate $r$ for discounting.


\subsection{Implied Volatility Surface Computation} \label{subsec:ImpliedVolaSurfaceComputation}

For each computed option price, the corresponding implied volatility is obtained by inverting the Black-Scholes pricing formula. Let $V_{\text{MC}}$ denote the Monte Carlo price of a call or put option with strike $K$ and maturity $T$. The implied volatility $\sigma_{\text{IV}}$ is defined as the unique positive solution to
\begin{equation}
    V_{\text{BS}}(K, T; \sigma_{\text{IV}}) = V_{\text{MC}},
\end{equation}
where $V_{\text{BS}}$ denotes the Black-Scholes price given the same $K$, $T$, $r$, and $S_0$.

The inversion is carried out numerically using a root-finding algorithm, such as the bisection method, with appropriate upper and lower bounds, and convergence tolerance. To ensure robustness, option values that are near zero or very deep in-the-money are excluded from inversion, as they may lead to numerical instability.

The result is a matrix of implied volatilities $\sigma_{\text{IV}}(K_l,T_q)$ that forms the simulated volatility surface. This surface serves as the primary output of the model and forms the basis for the analysis in Section~\ref{sec:SimulationResultsAnalysis}.


\subsection{Output for Subsequent Analysis} \label{subsec:OutputSubsequentAnalysis}

Each complete simulation run with $M$ Monte Carlo paths and fixed model parameters is referred to as a \emph{scenario}. A scenario is fully characterized by a specific configuration of the parameters $(J, \gamma_1, \gamma_2, \rho, a, b, r)$. The roughness parameter $H$ is not an independent input, but is determined by the relation $H = \gamma_1 + \gamma_2 - \frac{1}{2}$.

The output generated from a scenario consists of three main components derived from the simulated asset price paths and option pricing procedures.

First, two matrices of option values are computed: one for European call options and one for European put options. These matrices are evaluated over a common pre-defined grid of strikes $\{K_1, \ldots, K_L\}$ and maturities $\{T_1, \ldots, T_Q\}$. Both matrices are retained to allow for consistency checks, such as verification of put-call parity.

Second, the option prices are transformed into implied volatilities via numerical inversion of the Black-Scholes formula, as described in Section~\ref{subsec:ImpliedVolaSurfaceComputation}. This results in two implied volatility matrices, one for calls and one for puts. These are then aggregated by computing the element-wise average, wherever both values are available. If only one implied volatility value is available (due to numerical issues), that value is used directly.

The final output per scenario thus consists of the following elements: 
\begin{itemize}
    \item The list of scenario parameters $(J, \gamma_1, \gamma_2, \rho, a, b, r)$,
    \item Two option value matrices (calls and puts), primarily used for diagnostic purposes,
    \item One aggregated implied volatility matrix used in the subsequent analysis.
\end{itemize}

All matrices are computed on the same grid of strikes and maturities. This aggregated surface is the primary object used in the simulation-based analysis that follows.