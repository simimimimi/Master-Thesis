\section{Calibration to Market Data} \label{sec:Calibration}

This section validates the Hypergeometric Volatility Model by comparing its simulated implied volatility surfaces to observed market data. The analysis is based on S\&P 500 option data from the OptionMetrics database (accessed via WRDS) for March 2023. 

A mean implied volatility surface is constructed over a fixed moneyness-maturity grid using interpolation techniques. First, the simulation quality criteria from Section~\ref{subsec:SimulationQualityCriteria} are evaluated on the empirical surface. Then, the model is calibrated by identifying the parameter set that produces the best fit to the market surface in terms of Mean Squared Error (MSE).

\subsection{Empirical Volatility Surface}

\subsubsection*{Data Collection and Preparation}
S\&P 500 options data from March 2023 were retrieved via the WRDS interface to the OptionMetrics database. The dataset includes strike prices, expiration dates, and implied volatilities. Data cleaning was performed by removing invalid entries, including:
\begin{itemize}
    \item Implied volatilities exceeding 100\%
    \item Maturities exceeding 3 years
    \item Extreme moneyness values (very deep in- or out-of-the-money options)
\end{itemize}

The cleaned data were then standardized: strike prices were converted to moneyness using the ratio $K/S$, and maturities were converted from calendar days to years. The resulting data points were binned onto a discrete moneyness-maturity grid and averaged to reduce noise. A bilinear interpolation method was applied to construct a smooth implied volatility surface compatible with the model's simulation grid.

\subsubsection*{Evaluation of Quality Criteria}
The empirical surface satisfies all four simulation quality criteria that can be evaluated using implied volatility data. The put-call parity criterion cannot be tested directly, as the OptionMetrics database provides implied volatilities but not raw option prices. 

Figure~\ref{fig:market_surface} and the associated diagnostics (smile slices, ATM skew, and log-log skew term structure) confirm that the empirical surface exhibits smile convexity, negative ATM skew, an increasing skew term structure, and approximate power-law behavior.
\begin{figure}[H]
    \centering
    \includegraphics[width=0.45\textwidth]{figures/6.1.2 Market Surface/market_iv_surface.png}
    \includegraphics[width=0.45\textwidth]{figures/6.1.2 Market Surface/market_iv_smiles.png}
    \includegraphics[width=0.45\textwidth]{figures/6.1.2 Market Surface/market_atm_skew.png}
    \includegraphics[width=0.45\textwidth]{figures/6.1.2 Market Surface/market_atm_skew_log.png}
    \caption{S\&P 500 Volatility Surface from March 2023}
    \label{fig:market_surface}
\end{figure}


\subsection{Calibration}

\subsubsection*{Calibration Methodology}
The calibration procedure evaluates how well each scenario generated by the simulation framework matches the empirical volatility surface. For each parameter configuration, the model computes a simulated surface and compares it to the market surface point-by-point using the Mean Squared Error (MSE):
\begin{equation}
    \text{MSE} = \frac{1}{N} \sum_{i=1}^{N} \left( \sigma_{\text{model}}^{(i)} - \sigma_{\text{market}}^{(i)} \right)^2,    
\end{equation}
where $\sigma_{\text{model}}^{(i)}$ and $\sigma_{\text{market}}^{(i)}$ denote implied volatilities at grid point $i$, and $N$ is the number of grid points with valid data.

The parameter set yielding the lowest MSE is selected as the best fit to the market surface.

\subsubsection*{Best Fit Parameter Set}
The optimal parameter configuration minimizing the MSE is
\begin{equation*}
    \{ H = 0.20,\quad a = 1.0,\quad b = 0.20,\quad \gamma_2 = 0.20,\quad \gamma_1 = 0.50 \}.
\end{equation*}

Figure~\ref{fig:best_fit_surface} displays the resulting implied volatility surface, along with slices of volatility smiles, the ATM skew term structure, and the log-log power-law fit. Visual comparison with the empirical surface suggests a strong match, particularly in the short-maturity and near-the-money region.

These results demonstrate that the Hypergeometric Volatility Model can replicate key features of observed market volatility surfaces, provided the parameters lie within the empirically favorable ranges identified in Section~\ref{subsec:SuccessRateAnalysis}.
